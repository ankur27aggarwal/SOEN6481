\documentclass{article}
\usepackage[utf8]{inputenc}

\title{\centering  Choice of Interviewee and Interview Questions }
\author{\centering Ankur Aggarwal }
\date{July 2019}

\begin{document}

\maketitle
\section*{\textbf{ Criteria for the selection of the interviewee  :}}
\begin{flushleft}
\hrulefill
\end{flushleft}
\noindent As the Gaussian Integral is used for the normalization of the samples and for the calculation of the err function vividly . It has various usage in the field of electronics and computer graphics , to render the data in the normalized form . My focuss was for this problem was to find the interviewee how have been using the number for relevant amount of time . Also for whome generating a calculator will help them in their daily calculation with the number. Moreover using their experience how much further I can improve the calculator to incollaborator more functionality related to the calculator . 
\section*{\textbf{ Interviewee selected for the task :}}
\begin{flushleft}
\hrulefill
\end{flushleft}
\noindent For the detailed analasis of the questions I have choosen two interviewee from two different background , one is from the computer science and other is from the electronics background
\\
\\ \textbf{Interviewee 1 :}
\\ Name :- Richa Aggarwal 
\\ Background :- Electronics (M.S. integrated circuit and system from The University of Texas at Austin )
\\ Experience working with number :- Probably working from the past 2 and half year with the number .
\\
\\ \textbf{Interviewee 2 :}
\\ Name :- Sumit 
\\ Background :- Computer Science (M.S. integrated circuit and system from The University of Texas at Austin )
\\ Experience working with number :- working from approximately 3 year with the number .
\section*{\textbf{ Interview Question asked :}}
\begin{flushleft}
\hrulefill
\end{flushleft}
\noindent As per the requirment I have categorized my question into three different categories : 
\\ 1. General Question 
\\ 2. Number specific 
\\ 3. General to Calculator \newline \newline \newline \newline
\begin{center}
   \textbf{ General Question :}    
\end{center}
\begin{flushleft}
\hrulefill
\end{flushleft}
\noindent Interview Question 1 :- what is your name ?
\\Response:
\\  Interviewee 1 :- Richa Aggarwal
\\  Interviewee 2 :- Sumit  \newline \newline
\noindent Interview Question 2 :- How long have you been working with the Gaussian Integral  ?
\\Response:
\\  Interviewee 1 :- working from the past 2 and half year with the number .
\\  Interviewee 2 :- working from approximately 3 year with the number . \newline \newline
\noindent Interview Question 3 :- where do you use it ?
\\Response:
\\  Interviewee 1 :- In normalization of the binary data from the circuit and also in normalization of the signals to convert them into sine form .
\\  Interviewee 2 :- In the gussian filter for rendering the pixels in computer graphics \newline \newline
\noindent Interview Question 4 :- How often do you use it ?
\\Response:
\\  Interviewee 1 :- Once or Twice a month . 
\\  Interviewee 2 :- Approximately three to four times in a week  \newline \newline
\noindent Interview Question 5 :- How do you calculate the number  now ?
\\Response:
\\  Interviewee 1 :- Usually Online portal
\\  Interviewee 2 :- Company provide service  \newline \newline \newline


\begin{center}
   \textbf{ Number Specific Question :}    
\end{center}
\begin{flushleft}
\hrulefill
\end{flushleft}
\noindent Interview Question 1 :- Do you want your result in the form $\pi$ or in numerical format ?
\\Response:
\\  Interviewee 1 :- In the form of numbers .
\\  Interviewee 2 :- For changing the pixel ratio numerical format is the best . \newline \newline
\noindent Interview Question 2 :- For upto how many digit you want to calculate the result ?
\\Response:
\\  Interviewee 1 :- Approximation could be considered upto 4 digits .
\\  Interviewee 2 :- upto 2 or 3 digits .  \newline \newline
\noindent Interview Question 3 :- Do you want the limits of the integration to be fixed from ?
\\Response:
\\  Interviewee 1 :- Yes , as we wants to calculate the result for the large number of samples therefore limits over the entire real line
\\  Interviewee 2 :- No, sometimes limits change for small intervals\newline \newline
\noindent Interview Question 4 :- Do you want to combine the result of Gaussian Integer with some other integral or multiplier
\\Response:
\\  Interviewee 1 :- NO, we only requires the value of the gaussian integral other operations are done seperately  . 
\\  Interviewee 2 :- Yes , we have to multiply the number with aspect ratio to the better sampling of numbers.  \newline \newline















\end{document}
